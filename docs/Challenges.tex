\section{Challenges}

During the project's development, several challenges were faced, requiring organization, adaptability, and collaborative work from the team.

One of the main obstacles was finding a balanced way to divide tasks, ensuring that all members could actively participate without overburdening anyone. The modularization of the code and the clear definition of responsibilities required several discussions and adjustments throughout the process.

Another point of difficulty was the interpretation of the project statement. The text provided by the professor contained ambiguities in some requirements, which led to uncertainty during implementation. A collective effort was required to correctly interpret the guidelines and align the team's expectations with the project's objectives.

The second dataset used also posed a significant technical challenge. Due to its large size, processing the data completely demanded more computational resources than the team's personal computers could offer. As a consequence, we experienced frequent VSCode freezes and unexpected terminal crashes.

Creating the illustrations for the report also presented difficulties. We opted to use LaTeX as our scientific writing tool, which provided advantages in formatting and standardizing the document. However, creating graphs and visual representations, such as the trees, required the use of complex libraries like TikZ, which demanded a significant time investment to learn and apply correctly.

Initially, the tree printing function only displayed data in the terminal. As the project progressed, the need to save this information to files arose, which required modifying the code to allow for automatic output redirection.

Another challenge was visualizing the performance statistics. Although the team opted to use C++ for this task, the language lacks robust native support for generating graphs. It was therefore necessary to find specialized libraries and learn how to use them effectively.

Finally, as is common in projects of this nature, we encountered logic errors during implementation, especially during the tree construction and balancing phases. Identifying and correcting these flaws required careful code review, constant testing, and collaboration among the members to ensure the stability and efficiency of the data structures.