\section{Conclusion}

 Furthermore, it is observed that the patterns are consistent with those observed in both datasets.
 However, due to the larger number of nodes in this second dataset, the time differences between the approaches
 are more significant, confirming the importance of balanced structures in large-scale applications.

 Comparing the three structures, the BST stands out for its simplicity, but can have poor performance when data is inserted in an orderly fashion—becoming practically a list.
 The AVL tree keeps the tree always very well balanced, which guarantees fast searches, but requires more rotations, making it better for cases with many queries and few insertions.
 The RBT, on the other hand, finds a middle ground: it performs fewer rotations than AVL and still maintains good overall performance, making it a good choice when there are many insertions and modifications to the tree.