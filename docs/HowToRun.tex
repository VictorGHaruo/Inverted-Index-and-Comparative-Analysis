\section{How to Run}

This guide provides instructions for compiling and running the project on Linux systems or the Windows Subsystem for Linux (WSL).

\subsection{Prerequisites}

Ensure the following packages are installed on your system:
\begin{itemize}
    \item C++ Compiler (g++ 11.4.0 or later)
    \item \texttt{make}
    \item \texttt{unzip}
    \item Qt5 development libraries
\end{itemize}

\subsection{Setup}

\subsubsection{1. Install Dependencies}

On Debian-based systems like Ubuntu, run the following commands to update your package lists and install the required dependencies:

\begin{minted}[frame=single, bgcolor=lightgray!50]{bash}
# Update package lists
sudo apt update

# Install make and unzip
sudo apt install make unzip

# Install Qt5 development tools
sudo apt install qtbase5-dev qtchooser qt5-qmake qtbase5-dev-tools -y 
\end{minted}

\subsubsection{2. Prepare the Datasets}

The project datasets are provided as compressed \texttt{.zip} files. Due to how they are archived, unzipping them creates a nested directory structure (e.g., \texttt{data/data/}). The commands below will unzip the archives and move the files into their correct locations.

From the project's root \texttt{InvertedIndex/} directory, run:
\begin{minted}[frame=single, bgcolor=lightgray!50]{bash}
# Unzip the primary dataset into the data/ directory
unzip data/data.zip -d data/
# Unzip the secondary dataset into the data2/ directory
unzip data2/data_v2.zip -d data2/

# Move the files out of the nested 'data' directories
mv data/data/* data/
mv data2/data/* data2/
\end{minted}


\subsection{Compilation \& Execution}

This project uses a \texttt{Makefile} located in the \texttt{src/} directory to simplify compilation and execution. You can run the program with default settings or provide custom parameters.

\subsubsection{Using the Makefile}

The following commands should be run from the project's root directory.
    
    \begin{minted}[frame=single, bgcolor=lightgray!50]{bash}
#To run with default settings:
(cd scr/ ; make)

#To specify the number of documents (uses default BST tree):
(cd scr/ ; make num=n)
    
#To run with fully custom parameters:
(cd scr/ ; make tree=tree path="../data_folder/" num=num cmd=stats)

#Delete object files and executables: 
(cd scr/ ; make clean)
    \end{minted}



The following parameters can be passed to the \texttt{make} command to customize execution.

\begin{table}[H]
\centering
\renewcommand{\arraystretch}{1.3}
\begin{tabular}{|c|>{\centering\arraybackslash}p{6cm}|c|c|c|} 
\hline
\textbf{Parameter} & \textbf{Description} & \textbf{Default} & \textbf{Arguments} & \textbf{Example} \\
\hline
\texttt{tree} & Data structure to use & \texttt{bst} & bst, avl, rbt & \texttt{tree=avl} \\
\texttt{num} & Number of documents to process & \texttt{10103} & 1, 2, 3, ... & \texttt{num=1000} \\
\texttt{cmd} & Command to execute & \texttt{search} & search, stats & \texttt{cmd=stats} \\
\texttt{path} & Path to the documents folder & \texttt{../data/} & \texttt{../data/}, \texttt{../data2/} & \texttt{path=../data2/} \\
\hline
\end{tabular}
\caption{Makefile Parameters}
\label{tab:config_params}
\end{table}

\textbf{Note:} If parameters are not specified, the program will default to using the \textbf{BST} structure with all \textbf{10103} documents from the \texttt{data/} folder and will execute the \texttt{search} command.

\subsection{Plot Graphs}


The graphs will be created on the plots folder.

    \begin{minted}[frame=single, bgcolor=lightgray!50]{bash}
#Compiling files to plot graph
(cd plots && qmake && make && ./plot)
    \end{minted}

