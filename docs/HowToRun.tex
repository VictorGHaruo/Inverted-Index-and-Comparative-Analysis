\section{How to Run}

This guide provides instructions for compiling and running the project on Linux systems or the Windows Subsystem for Linux (WSL).

\subsection{Prerequisites}

Ensure the following packages are installed on your system:
\begin{itemize}
    \item C++ Compiler (g++ 11.4.0 or later)
    \item \texttt{make}
    \item \texttt{unzip}
    \item Qt5 development libraries
\end{itemize}

\subsection{Libraries}
\begin{itemize}
    \item <iostream>
    \item <vector>
    \item <fstream>
    \item <string>
    \item <chrono>
\end{itemize}

\subsection{Setup}

\subsubsection{Install Dependencies}

On Debian-based systems like Ubuntu, run the following commands to update your package lists and install the required dependencies:

\begin{minted}[frame=single, bgcolor=lightgray!50]{bash}
# Update package lists
sudo apt update
sudo apt upgrade

# Install make and unzip
sudo apt install make unzip

# Install Qt5 development tools
sudo apt install qtbase5-dev qtchooser qt5-qmake qtbase5-dev-tools -y 
\end{minted}

\subsubsection{Prepare the Datasets}

The project datasets are provided as compressed \texttt{.zip} files. Due to how they are archived, unzipping them creates a nested directory structure (e.g., \texttt{data/data/}). The commands below will unzip the archives and move the files into their correct locations.

From the project's root \texttt{InvertedIndex/} directory, run:
\begin{minted}[frame=single, bgcolor=lightgray!50]{bash}
# Unzip the primary dataset into the data/ directory
unzip data/data.zip -d data/
# Unzip the secondary dataset into the data2/ directory
unzip data2/data_v2.zip -d data2/

# Move the files out of the nested 'data' directories
mv data/data/* data/
mv data2/data/* data2/
\end{minted}


\subsection{Compilation \& Execution}

This project includes a \texttt{Makefile} in the \texttt{src/} directory to streamline compilation and execution. The program can be run with custom parameters, either manually or automatically via the \texttt{Makefile}.


\begin{minted}[frame=single, bgcolor=lightgray!50]{bash}
#Compile all the files to run the trees:
(cd scr/ ; make)
\end{minted}

\subsubsection{Running Manually}
The following commands should be run from the project’s root directory.
\begin{minted}[frame=single, bgcolor=lightgray!50]{bash}
#Compile all the files to run the trees (*substitute tree_name with a tree name):
(cd scr/ ; ./main_tree_name <command> <n_docs> <directory_path>)

#Example
(cd scr/ ; ./main_bst stats 10 ../data2/)

\end{minted}

\subsubsection{Running Using the Makefile}

The following commands should be run from the project's root directory.
    
    \begin{minted}[frame=single, bgcolor=lightgray!50]{bash}
    
#To run with Makefile:
(cd scr/ ; make tree=tree path="../data_folder/" num=num cmd=stats)

#Example:
(cd scr/ ; make tree=bst path="../data2/" num=100 cmd=stats)


#Delete object files and executables: 
(cd scr/ ; make clean)

#Delete all object(data, i.e .CSV .txt) files and executables created by the code.
(cd scr/ ; make cleanall)
    \end{minted}

\textbf{Additionally:} When you run the code, with stats command there a few option to run:

\begin{enumerate}
    \item Quit and Save the Stats in a csv.
    \item Print the Tree
    \item Save the Print in a ".txt"
    \item Print Index 
    \item Save the Printed Index
\end{enumerate}


\subsection{Parameters}
The following parameters can be passed to the \texttt{make} command and the manual way to customize execution.

\begin{table}[H]
\centering
\renewcommand{\arraystretch}{1.3}
\begin{tabular}{|c|>{\centering\arraybackslash}p{6cm}|c|c|c|} 
\hline
\textbf{Parameter} & \textbf{Description} & \textbf{Arguments} & \textbf{Example} \\
\hline
\texttt{tree} & Data structure to use & bst, avl, rbt & \texttt{tree=avl} \\
\texttt{num} & Number of documents to process & 1, 2, 3, ... & \texttt{num=1000} \\
\texttt{cmd} & Command to execute & search, stats & \texttt{cmd=stats} \\
\texttt{path} & Path to the documents folder & \texttt{../data/}, \texttt{../data2/} & \texttt{path=../data2/} \\
\hline
\end{tabular}
\caption{Makefile Parameters}
\label{tab:config_params}
\end{table}

\textbf{Note:} The default command is "stats". If you don't specify the number of files, all files will be used by default. Make sure to provide the correct file path.

\subsection{Plot Graphs}


The graphs will be created on the plots folder.

    \begin{minted}[frame=single, bgcolor=lightgray!50]{bash}
#Compiling files to plot graph
(cd plots && qmake && make && ./plot)
    \end{minted}

\textbf{Note:} To plot the graphs, you must have the .csv file containing the data. You can generate this file by running the tree code in stats mode and selecting the first option.

\subsection{Tests}

\subsubsection{Running Using Makefile}
\begin{minted}[frame=single, bgcolor=lightgray!50]{bash}
#Compile all the files to run the all trees tests:
(cd scr/ ; make test)

#Run the tests for tree_name (*substitute with tree name)
(cd scr/ ; make test tree=tree_name)

#Example
(cd scr/ ; make test tree=bst)
\end{minted}

\subsubsection{Running Manually}

\begin{minted}[frame=single, bgcolor=lightgray!50]{bash}
#Compile all the files to run the all trees tests:
(cd scr/ ; ./test_name_tree)

#Example
(cd scr/ ; ./test_bst)
\end{minted}

