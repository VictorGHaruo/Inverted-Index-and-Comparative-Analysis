\section{What Each One Has Done}

Throughout the project's development, all team members remained consistently active, as evidenced by the commit history. Each member contributed to both discussions and coding. To optimize the workflow, the project was modularized, allowing specific parts to be primarily developed by certain members—albeit with collective support and review.

To ensure the application of knowledge acquired in Data Structures and Algorithms, each member actively participated in implementing at least one tree structure. Development began with the common auxiliary functions for all trees, such as node creation, printing, and property calculation, which were implemented by Rodrigo Severo.

The project is founded on the Binary Search Tree (BST), which serves as the basis for the other trees. Accordingly, the implementation of the BST, along with the initial project structure, was carried out by Victor Iwamoto. He developed the core tree functions, such as search, destroy, and create.

For the self-balancing trees, Everton Reis and Lucas Menezes jointly developed the entire logic for the AVL Tree, implementing the necessary rotations to maintain its balance during insertions.

Given the greater complexity of the Red-Black Tree, the pair programming technique was adopted to ensure collaborative development and higher quality control. In this process, Rodrigo Severo acted as the driver (coder), while Eric Ribeiro served as the navigator (strategist and reviewer).

Furthermore, Eric Ribeiro was also responsible for the user experience design of the command-line interface (CLI). He mapped out potential user errors and implemented appropriate handling mechanisms. To improve overall usability, he created the project's \textit{Makefile} and developed the data reading system.

Everton Reis was in charge of the internal documentation of the source code. This required him to review and understand the modules developed by the other members, which meant he also acted as a technical reviewer.

Lucas Menezes focused on the project's statistical analysis, developing graphs from the trees' performance metrics using a C++ library specialized in data visualization.

Based on the statistics generated by the CLI and the graphical visualizations, Rodrigo Severo wrote the project's conclusions, comparing the practical performance of the structures and highlighting their respective advantages and limitations.

Finally, Victor Iwamoto was responsible for drafting the final report, in which he detailed the developed structures and functions. This task required a complete review of the code, allowing him to also take the lead on quality assurance, identifying bugs and relaying them to their respective authors for correction. He was also in charge of the report's visual identity and final formatting.